% des paquetages indispensables, qui ajoutent des fonctionnalites
%\usepackage[latin1]{inputenc}
\usepackage[utf8]{inputenc}
%\usepackage[french]{babel}
\usepackage[english]{babel}
%\usepackage{lmodern}
\usepackage{amsmath,amssymb}
\usepackage{amsthm}
\usepackage{fullpage}
\usepackage{graphicx}
\usepackage{url}
\usepackage{xspace}
\usepackage{listings}
\usepackage{xcolor}
\usepackage{hyperref}
\usepackage{eurosym}
\usepackage{hyperref}

%%configuration de listings pour l'affichage du code
%\lstset{
%    language=C,
%    basicstyle=\ttfamily\small, %
%    identifierstyle=\color{red}, %
%    keywordstyle=\color{blue}, %
%    stringstyle=\color{black!60}, %
%    commentstyle=\it\color{black!50}, %
%    columns=flexible, %
%    tabsize=3, %
%    extendedchars=true, %
%    showspaces=false, %
%    showstringspaces=false, %
%    numbers=left, %
%    numberstyle=\tiny, %
%    breaklines=true, %
%    breakautoindent=true, %
%    captionpos=b,
%    frame=single
%}

% On fera des listes a puce et non a tiret.
%\renewcommand{\FrenchLabelItem}{\textbullet}

\newtheorem{theoreme}{Th\'{e}or\`{e}me}
\newtheorem{definition}{D\'{e}finition}
\newtheorem{exercice}{Exercice}

\newcommand{\tab}{\hspace*{\parindent}}

%Definition de quelques commandes utiles en maths :

% R et N
\newcommand{\R}{\mathbb{R}}
\newcommand{\N}{\mathbb{N}}
\newcommand{\C}{\mathbb{C}}
% derive partielle et congruence
\newcommand{\drond}{\partial}
\newcommand{\congru}{\equiv}
% blocs parenthese, valeur absolue, norme, crochets, accolades
\newcommand{\abs}[1]{\left\lvert#1\right\rvert}
\newcommand{\norm}[1]{\left\lVert#1\right\lVert}
\newcommand{\braces}[1]{\left(#1\right)}
\newcommand{\croch}[1]{\left[#1\right]}
\newcommand{\cbraces}[1]{\left\{#1\right\}}
% blocs parties entieres superieur et inferieur
\newcommand{\entsup}[1]{\left\lceil#1\right\rceil}
\newcommand{\entinf}[1]{\left\lfloor#1\right\rfloor}
